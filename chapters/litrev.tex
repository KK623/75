\chapter{The separation of the social and the linguistic}
\label{ch:litrev}

\epigraph{It's kind of like there's youth culture and then there's human beings and it's really nice to be like accepted as a human being.}{Katrina (The Relaxed Group). Interview, 18-10.}


\section{Introduction}

High school can be a difficult period as it marks the transition between childhood and adulthood. Adolescents are expected to take on additional responsibilities but are not yet treated like adults or, as Katrina expressed feeling, not yet treated like human beings. This transitional period is marked by linguistic variation, as the teenagers ``try on'' different personae in an effort to construct their identities within the context of the changing perceptions of their identities.

Additionally, there is pressure from within the social make-up of the school, where an individual's style is often interpreted as a reflection of who she is \citep[2]{pomerantz2008}. While \citet{pomerantz2008} focused on clothing styles, this is true of other aspects of an individual's style, where style is defined as a ``socially meaningful clustering of features, within and across linguistic levels and modalities'' \citep{campbellkibleretal2006} and non-linguistic levels and modalities. High school students construct their identities in relation to each other (in addition to the world around them) and in doing so, they make use of a multitude of stylistic components, including ways of dressing, ways of walking, and ways of talking. 

In this book, I examine the link between linguistic variation and identity in order to develop our understanding of the ways in which language and social information are stored in the mind and accessed during the production and perception of speech. Specifically, I examine the degree to which lemma-based phonetic variables are manipulated in the construction of social personae and I investigate the extent to which the relationship between social, phonetic, and lemma-based information influences speech processing. Within the context of data from an all girls' school, I argue that social theory needs to be incorporated into linguistic theory and in \chapref{ch:disc}, I present a possible avenue in which to explore this unification of theories.

Along with \citet{weinrichlabovherzog1968}, I believe that a

\begin{quote}
	nativelike command of heterogeneous structures is not a matter of multidialectalism or ``mere'' performance, but is part of unilingual linguistic competence \citep[101]{weinrichlabovherzog1968}.
\end{quote}

\noindent Empirical evidence can bring to light the richness and complexity of this competence, resulting in a better understanding of linguistic patterns found at all levels of the grammar. Using empirical methods to inform a unified probabilistic model of identity construction, speech production, and speech perception, the research questions I explore here relate both to social theory and to how social information is stored in the mind and is indexed to linguistic representations. The specific questions to be addressed are:

\begin{enumerate}
	\item Can lemmas that share a wordform have different realisations? 
	\item Do speakers manipulate their realisations of a lemma in the construction and expression of their identity? 
	\item What is the relationship between the phonetic realisation of a lexical item and how predictable that item is given who the speaker is?
	\item How is this construction of personae related to other speakers who share a similar stance?
	\item And what role does this phonetic, lemma, and social information play during speech processing?  
\end{enumerate}

\noindent In order to address these questions, I have employed the use of multiple methodologies within a single study, combining the qualitative method of ethnography with the quantitative methods of acoustic analysis and experimental design. 

I spent a year at Selwyn Girls' High, the pseudonym for the all girls' high school in Christchurch, New Zealand where I chose to conduct an ethnographic investigation of identity construction. The girls shared details of their lives with me and allowed me to record their conversations. While there were a number of close-knit groups at the school, these groups could be categorised according to whether they embodied, created, and perpetuated the school's norms (forming what I refer to as Common Room groups) or whether they dismissed, rejected, or failed to conform to these norms (forming what I refer to as non-Common Room groups). The qualitative findings from the ethnography are presented in \chapref{ch:ethnography}.

The linguistic analysis focuses on the word \textit{like}, a word with a number of different functions including the quotative (\textit{and Mum's \textbf{like} ``turn that stupid thing off''}), the lexical verb (\textit{I don't really \textbf{like} her that much}), and the discourse particle (\textit{Lily was \textbf{like} checking out my brother}). In \chapref{ch:prod}, I discuss the frequency with which different girls and groups at the school used these different functions and I present results from acoustic analysis conducted on tokens of \textit{like} from the girls' speech. I discuss the results within the context of theories of identity construction and consider the possibility that colloquial words can serve as loci for socially-meaningful phonetic variation. The work presented in \chapref{ch:prod} can be found in article-form in \citet{drager2011-JPhon}.


In \chapref{ch:perc}, I present the method and results from three perception experiments that I conducted at the school, which are also presented in \citet{drager2010-LabPhon}. The experiments were designed with the aim of determining whether perceivers could use phonetic cues in the signal to identify a word (here, a particular function of \textit{like}) and whether they could extract social information attributed to a speaker when exposed to only short clips of speech that contain phonetic and lemma-based information. 

In \chapref{ch:disc}, I discuss the results within the context of two linguistic models: one that relies on Bayesian statistics \citep{jurafsky1996,narayananjurafsky2002} and an exemplar model of speech production and perception, where complete acoustically-detailed representations of encountered utterances are stored in the mind \citep{johnson1997,pisoni1997,pierrehumbert2001}. I then argue for the need to incorporate theories of identity construction into linguistic models and I propose a model in which to explore this unification. In the concluding chapter, I discuss some developments in the field since writing my dissertation.

In order to inform the presentation of methods and results in the following chapters, the remainder of this chapter reviews relevant literature, focusing on the development of social theory within linguistics, recent insights into the storage of sociophonetic relationships in the mind, and other work which demonstrates the probabilistic nature of linguistic variation. This book is an adaptation of my Ph.D. dissertation \citep{drager2009-thesis}. Therefore, much of the background literature I discuss is reflective of the field at that time though I have added some discussion of more recent work, when needed. Additionally, I have added a concluding chapter that discusses an avenue for future exploration of work along these lines.

						
\section{The social, the linguistic, and the cognitive}

\epigraph{I have resisted the term \textit{sociolinguistics} for many years, since it implies that there can be a successful linguistic theory or practice which is not social.}{\citet[xix]{labov1972sociolingpatterns}}

\noindent Despite the fact that language use occurs in a social realm, sociolinguistic findings are rarely incorporated into formal linguistic models; socially-conditioned linguistic variation has been treated as an epiphenomenon to grammatical and phonological variation. This tendency had its beginnings over a century ago with Saussure's distinction between \textit{langue} (the knowledge of a language's structure that is shared across the speakers of that language) and \textit{parole} (the actual language used by an individual in their everyday life) \citep{saussure1916}. Saussure believed that \textit{langue}, with its regularity and structure, should be the focus of linguistic study and that \textit{parole} was too erratic and variable to be of scholarly interest. Half a century later, \citet[4]{chomsky1965} built on this with the distinction between \textit{competence} (a speaker-hearer's knowledge of his or her language) and \textit{performance} (actual language use in everyday life), later making the differentiation between I-language (internalised language) and E-language (externalised language) \citep[20-22]{chomsky1986}. The focus of structural linguistic theory has been \textit{langue}, competence, and I-language, treating language as invariant and linguistic categories as absolute. Methodologies used to investigate internalised linguistic structure typically include eliciting data from a native speaker of a particular language or relying on the intuitions of the researcher. Surveys are also sometimes conducted, while other studies use texts to determine whether certain structures are grammatical. 

In attempting to answer the question of \textit{how language works}, it is imperative that social effects on linguistic structure be investigated. This cannot occur only by studying the homogeneous linguistic knowledge of an ``ideal'' speaker-hearer, nor can it occur only by investigating the relationship between linguistic variation and broad social categories. Language is both social and individualistic; the construction of a symbol's meaning is a social enterprise and how this information is stored and used by a speaker-hearer is determined both by the unique experiences of that individual and by the experiences shared with others from the same community. In an investigation of identity, researchers must study both the community and the individual, ultimately examining the relationship between them \citep[146]{wenger1998}. Similarly, language does not belong only to an individual or only to the society to which that individual belongs; language exists within and across both. Linguistic variation that in Saussure's time was considered too messy to be investigated is now known to correlate with a number of factors, including social characteristics of the speaker and the formality of the situation \citep{labov1972sociolingpatterns}, token frequency (the number of times a speaker has encountered a word) \citep{bybee2002}, and how predictable a word is given its position in a sentence \citep{jurafskyetal2002}. Furthermore, there is evidence that this information is stored and affects speech processing \citep{strand1999,jurafsky2003}. Variation is not somehow systematic ``noise'' that is filtered out; it is stored and used during the perception and production of speech.

Sociolinguists have made \textit{parole}, performance, and E-language the focus of their investigation, examining the large amount of variation across different speakers and within the speech of a single individual. While there is a great deal of variation, much of it is predictable based on social characteristics of the speaker, the persona that the speaker is constructing in a given situation, and the various stances a speaker takes during an interaction. The variation is not only predictable but meaningful; it is a component of linguistic knowledge. Researchers examining this variation argue that a speaker's communicative competence is reflected in their behaviour \citep{hymes1972}. Therefore, examining this behaviour (i.e.~actual language in use) provides insight into how language is stored in the mind and accessed during speech production and perception. 

Empirical methods of linguistic study allow researchers to ``avoid the inevitable obscurity of texts, the self-con\-scious\-ness of formal elicitations, and the self-decep\-tion of introspection'' \citep[xix]{labov1972sociolingpatterns}. Empirical methods provide a means of examining speakers' behaviour with the intention of identifying patterns among the variation. Traditionally in the investigation of sociophonetic patterns, these methods involve the quantitative analysis of variables from sociolinguistic interviews (see \S \ref{interview:method}), but a growing number of studies use experimental methodologies (see \S \ref{sec:perception}). Both methods help demonstrate how linguistic variation is dependent on both social and linguistic information.


Outside of sociolinguistics, there is a growing body of work by researchers who use empirical methods to examine language in use \citep{bodetal2003}. Like sociolinguists, they have made gradient ``messy'' variation the focus of their research and have shed new light on the nature of the variation. This work provides strong evidence that language (at all levels of the grammar) is probabilistic; there is a great deal of variation in language and it is predictable if treated stochastically.\footnote{I would not argue that the study of language based on intuitions has no place in linguistics. However, I do believe that this method can only come part-way in answering the multitude of questions that ultimately address how language works.}

Insights into how language is stored and accessed during production and perception can be gained by investigating: \nocite{saussure1916}  

\begin{enumerate}
	\item how language is used in everyday life across different speakers, by individual speakers, and at all levels of the grammar; and
	\item how perceivers are influenced by trends from production based on both linguistic and non-linguistic information.
\end{enumerate}

\noindent Patterns in the production and perception of speech, regardless of whether they are conditioned by linguistic or social factors, can tell us something about a speaker's linguistic competence, blurring the traditional boundaries between competence and performance, \textit{langue} and \textit{parole}. 



In this chapter, I present research that has informed the work presented in this book. Because I used a number of methods (ethnography, acoustic analysis, and experimental design) and I address a number of theoretical issues (the role of gradience, speaker-specific probability of producing a word, accessing the lemma versus the wordform, and the construction of an individual's identity), this requires stepping through a vast amount of work from traditionally distinct linguistic subfields. I begin by discussing the progression of social theory through the waves of variationist studies. I then describe results from sociophonetic work that uses acoustic analysis and I discuss how this challenges some key assumptions made by popular linguistic theories. Next I present findings from speech perception experiments that investigate the relationship between linguistic and non-linguistic information. At this point, the discussion digresses from work in sociophonetics and focuses on two questions of interest that (at the time of writing my dissertation) had largely not been addressed in the sociolinguistic literature, namely the degree to which token frequency influences phonetic realisations and the degree to which different words that share a wordform can have different realisations. 
						   
\section{Waves of variationist studies}\label{sec:waves}

Different speakers produce different realisations from one another and at least some of this variation is correlated with the speakers' social characteristics. Here I step through what \citet{eckert2005} refers to as the First, Second, and Third Waves of variation studies. 

Research in the First Wave treats social variables as indexed directly to broad social categories, such as age, gender, and socioeconomic status. Research in the Second Wave examines variation that is correlated with locally constructed social categories and research in the Third Wave treats linguistic variables as indexed both to a speaker's style and to a speaker's stance, ``a socially recognised disposition'' \citep[2]{ochs1990}.  

In addition to different views about the nature of indexation between linguistic and social factors, particular methodologies are associated with each wave.  In order to elicit data, researchers working in the First Wave use either quick and anonymous questions or standard sociolinguistic interviews; the researcher need not be highly familiar with the subjects to determine a correlation between a broad social category and a linguistic variable. Research in the First Wave focuses on participants who are assumed to be linguistically typical of a predetermined social category \citep[35]{milroy1987}.

In contrast, work in the Second Wave uses qualitative methodologies to determine locally constructed social categories that are meaningful to the speakers. For work in the Second Wave,	``the unit of study is the \textit{pre-existing social group}, rather than the individual as the representative of a more abstract social category'' (\cite[35]{milroy1987}, italics in original). A key tool for work in the Second Wave has been ethnography, a methodology adopted from anthropology that uses participant observation and qualitative analysis to describe a culture or group. It is useful for sociolinguists because linguistic variation is only one symbolic tool individuals can use to express their identities; there are a multitude of other symbols at work at any given time, each potentially unique within a given culture or community.\footnote{Here, the word \textit{symbol} refers to a linguistic or non-linguistic ``social object used for communication to self or for communication to others and to self'' \citep[42]{charon1995}.} In order to interpret the meaning behind the linguistic symbols, one must understand the context in which that symbol has meaning \citep[22]{savilletroike1982}. This is demonstrated by \citegen{labov1963} work on Martha's Vineyard, where speakers adopted local phonetic realisations associated with covert prestige (prestige associated with locally-based models) rather than those associated with overt prestige (prestige associated with externally-based models, often spoken by an influential group). Crucial to understanding this choice in variants was an understanding of the emotions and opinions of the people on the island. The inhabitants of the island had negative feelings toward the mainlanders who visited the island every summer. Rather than adopting the prestige forms produced by the visitors, a number of Martha's Vineyeard's inhabitants adopted variants produced by the local fishermen. 

Another method used for research in the Second Wave is Lesley Milroy's \textit{snowball} technique, where the researcher uses the social networks within a community to recruit new participants \citep[32]{milroygordon2003}. Though primarily a method of subject recruitment, the snowball technique can be used to study the speakers' social networks as an analytical construct as opposed to focusing on a social category. This method can facilitate qualitative analysis because of its focus on friendship ties; the researcher has access to more information about the speaker than would be gleaned from an interview with someone whose connections within a community were unknown. This method is unlike ethnography in that it does not necessarily involve extensive observation of individual speakers. While a fieldworker may choose to become more involved with a community of networks, this involvement is a key component of an ethnographic approach. 

In order to investigate the relationship between linguistic variants and a spea\-ker's style, work in the Third Wave employs qualitative methodologies like those used in the Second Wave. Style is made up of smaller components, such as the use of a certain word or a particular realisation of a vowel and these components are socially-meaningful; the styles and their meanings are co-dependent and constantly shifting. As the stylistic components are manipulated in different ways to construct an individual's style, they take on new meanings. An individual's style does not stem only from the manipulation of linguistic variants but also relies on non-linguistic factors, such as wearing certain clothes, walking a particular way, or adopting a specific posture. The combination of all of these factors, linguistic and non-linguistic, determine an individual's style. Therefore it is necessary for researchers working in the Third Wave to utilise qualitative methodologies such as ethnography to observe these styles, the styles' linguistic and non-linguistic components, and the components' constantly shifting meanings.

The names of each wave refer to the progression of social theory within sociolinguistics rather than to a strict linearity on a temporal scale. For example, although \citegen{labov1963} study on Martha's Vineyard predates his (\citeyear{labov1966}) study in New York, the New York study is considered First Wave while the Martha's Vineyard project is a key example of work in the Second Wave. In fact, the vast majority of work conducted today continues to be in the vein of the First Wave, its appeal no doubt stemming from the ability to gain insights in less time and with less emotional involvement than that imposed by methodologies used in the Second and Third Waves.\footnote{Ethnography and other methodologies that require repeated interactions between a subject and a researcher take a great deal of time and they can be emotionally exhausting. ``For the fieldworker such [Second Wave] studies are extremely demanding in energy, persistence, time and emotional involvement'' \citep[79]{milroy1987}.}	 In the following sections I step through examples of work conducted in the First, Second, and Third Waves of variation studies.

\subsection{First Wave}

Beginning with his seminal work on sociophonetic variation in both Martha's Vineyard and New York (Labov 1963, 1966), Labov has been the single most influential researcher in the field of sociolinguistics. In New York, \citet{labov1966} surveyed retail workers at three different department stores that each had target clientele from different socioeconomic groups. He demonstrated how realisations of \textipa{/r/} in the phrase \textit{fourth floor} patterned depending on the expected socioeconomic status of the addressee. Since then, a multitude of studies have arisen displaying trends in other languages and dialects, the majority of which have focused on phonetic variation that patterns with a group's social category (e.g., \cite{trudgill1972,romaine1978,wolfram1974}. 

The vast majority of sociophonetic work conducted on New Zealand English has been (and continues to be) in the vein of the First Wave. Some examples on New Zealand English (NZE) include work by \citet{maclagangordonlewis1999}, \citet{haymaclagan2010}, \citet{dalywarren2001}, and \citet{starksreffell2006}.


\subsection{Second Wave}\label{sec:secondwave}
Work in the First Wave demonstrates how linguistic variables are correlated with a speaker's social characteristics; the indexation between them is treated as direct. Through adopting an ethnographic approach, work in the Second Wave expands on the observation that linguistic variation is related to a speaker's social characteristics by focusing on the motivation behind the variation: why do certain groups adopt certain variants and avoid others? While, for example, \citet{trudgill1972} reflected on the possible motivations, these interpretations did not stem from observation of, or interaction with, the speakers themselves; they were based on observations of society more generally. In addition to providing a means of observing the meanings behind the variation, ethnography allows researchers to avoid using predetermined social categories, instead investigating social categories that are created by, and relevant to, the speakers themselves. In addition to Labov's work on Martha's Vineyard, studies in the Second Wave include work by \citet{holmquist1985}, \citet{eckert2000}, and \citet{milroymilroy1978}.  

Of these studies, the one which has most influenced the work presented here was conducted by \citet{eckert1989,eckert2000}. Employing an ethnographic approach at Belten High, a high school in a Detroit suburb, \citet{eckert1989,eckert2000} found that pho\-ne\-tic realisations in an individual's speech patterned with whe\-ther that individual was ca\-tegorised as a Jock or a Burnout, categories that were not based on social groups (tight groups of friends who, if asked, would name each other as part of a group) but on social network clusters (affiliations between individuals, not all of whom considered each other friends, but who nonetheless shared practices) \citep[11]{eckert2005}. Students were highly aware of these polarised categories and they applied the labels to males and females and to themselves and others. Where the Jocks took part in school activities and behaved largely as was expected by the school, the Burnouts rejected the school's expectations and were viewed as rebellious in the eyes of the school, smoking cigarettes and going ``cruising'' (driving in a car with friends without a predetermined destination). While the Jocks accepted the school's norms and strove for upward mobility, the Burnouts rejected the norms and valued cooperative peer networks. Eckert examined a number of variables that were undergoing change as part of the Northern Cities Shift \citep{labovyaegersteiner1972}. Although the change was most advanced in the city, it was evident in the speech of some Belten High students. Eckert found that, in addition to a correlation between phonetic realisation and being a Jock or a Burnout, the phonetic variables were related to each group's distinct construction and expression of femininity and masculinity. For example, Burnouts were more likely than Jocks to raise the nucleus of the diphthong \textipa{/ai/} (as in \textit{price}). But within each group, males and females behaved differently; Burnout girls produced a greater number of innovative variants than Burnout boys and Jock boys produced a greater number than Jock girls. Eckert argues that in developing patterns of behaviour, people orient to their own gender group within the context of the larger networks with which they are involved \citep[122-123]{eckert2000}. While traditional notions of femininity may have applied to Jock girls, they did not apply to Burnout girls; individuals in the different networks adopted socially-meaningful variables that expressed their membership as a Jock or a Burnout within the context of their own gender group. 

%Fought's (1999) ethnographic study at a school in California examined \textipa{/u/}-fronting and gang member status among Latinos. She found that individuals with no gang affiliation were more likely to produce tokens of \textipa{/u/} that were fronted, a variable found in the speech of Anglos in the area. This tendency for the variable to be associated with non-gang members was found for both male and female speakers, but the effect of socioeconomic status affected groups of males and females differently. Gang-affiliated females from middle socioeconomic backgrounds were more likely to produce fronted tokens than gang-affiliated females from lower socioeconomic backgrounds, and females with lower socioeconomic statuses and no gang-affiliation were speakers who produced some of the most-fronted tokens. Among males, however, only the non-affiliated middle class individuals produced fronted tokens. Fought argues that the patterns observed were a result of societal pressures from within the Latino community. Females were expected to be `good', causing women from lower socioeconomic backgrounds who wanted to mark their non-gang status to adopt variables associated with the middle class group. In contrast, men were expected to be `tough', a trait associated both with gang membership and with working class status, making it more difficult for men from lower socioeconomic groups to distinguish themselves linguistically as non-gang members.

Through obtaining an understanding of the local relations, values, and ideologies of a community, studies in the Second Wave gain insight into why phonetic variables are correlated with social group membership. But one key to understanding the variations remained: understanding the motivations of an individual. Do individuals orient to broad and local social groups (e.g., female and Jock), or are individuals' social goals more malleable and varied than these categories would imply? 


\subsection{Third Wave}\label{sec:thirdwave}

Where studies in the First and Second Waves view sociolinguistic variables as indexed to a social group, studies in the Third Wave treat stylistic practice as fundamental. Studies in the Third Wave examine how linguistic variants contribute to an individual's collection of styles and the construction of their social personae; they focus on social meaning where social meaning is not defined through membership in a social group but through the individual's stance and the expression of who they are. Variables and social categories are indexed indirectly through their direct relationship with style. This insight helps to explain how what the individual does in their everyday conversations (micro) manifests as socially-conditioned at the group level (macro).

Central to the Third Wave has been the investigation of a community of practice, a term coined by \citet{lavewenger1991} which Eckert and McConnel-Ginet define as ``an aggregate of people who, united by a common enterprise, develop and share ways of doing things, ways of talking, beliefs, and values - in short, practices'' \citep[186]{eckertmcconnellginet1999}. \citet{wenger1998} states that to be a community of practice, a group must be involved in mutual engagement, a joint enterprise, and a shared repertoire of practices. It is through these that a community of practice negotiates the meaning of the practices themselves, drawing on and connecting meaning to what people know and do not know \citep[73-85]{wenger1998}. Linguistic variables are adopted and rejected on the basis of this social knowledge, making communities of practice promising groups in which to observe socially-conditioned phonetic variation:


\begin{quote}
The individual constructs an identity --- a sense of place in the social world --- in balancing participation in a variety of communities of practice, and in forms of participation in each of those communities. And key to this entire process of construction is stylistic practice. \citep[17]{eckert2005}
\end{quote}

\noindent Thus, speakers create their own distinctive personae through combining linguistic variables (e.g., phonetic variants, lexical items, and syntactic constructions) and non-linguistic factors (e.g., clothing, make-up, and ways of walking) and these personae are located within a larger social order. Viewing her work at Belten High within the context of the Third Wave, \citet{eckert2005} described how the Jocks and Burnouts were in fact indexing stances through their use of both linguistic variables (e.g., the diphthong \textipa{/ai/}) and non-linguistic factors (e.g., cruising). Jocks were school-oriented and aimed for upward social mobility; Burnouts were neighbourhood-oriented and valued solidarity. Whereas the Jocks viewed the Burnouts as irresponsible and antisocial, the Burnouts viewed the Jocks as disloyal and status-oriented. \noindent The stances of these two communities of practice were diametrically opposed, and the observed patterns for the phonetic variables reflected this.

Among younger students, \citet{eckert1996nailpolish} identified linguistic variants that co-varied with non-linguistic cues such as nail polish, lip gloss, hair style, and new ways of walking. All of these cues served a symbolic means. Through adopting a socially-meaningful variant (e.g., backed \textipa{/\ae/} before a nasal) and taking part in certain activities (e.g., wearing nail polish), the girls each constructed an individual style that was to define their social persona. Other studies in the Third Wave include work by \citet{mendozadenton2008}, \citet{zhang2005}, and \citet{podesva2011}.

%\citet{mendozadenton2008} employed an ethnographic approach when studying the speech of Chicana and Mexicana gang girls in California. Mendoza-Denton found variation in realisations of the vowel /\textipa{I}/; girls who were central members of the gangs were more likely to produce raised variants of the vowel than girls who were \textit{wannabes}, ``groups of young people who participate in the symbolic display of gang culture... but have little to do with any committed aspects of gang affiliation'' \citep[57]{mendozadenton2008}. She also found that different girls wore varying amounts of eyeliner: a longer line was an indication that a girl was a more central member of the gang. In other words, girls who wore longer lines for their eyeliner were the same girls who produced raised variants of /\textipa{I}/; this co-variation between these two stylistic components demonstrates how both language and make-up were used by the girls to construct their identities and display their relation to the gang.

%\citet{zhang2005} employed an ethnographic methodology in Beijing and found that individuals' realisations of certain variants could be predicted by whether they worked for a government-owned company or were a ``Chinese yuppie'', someone who wore foreign brands of clothes, spoke foreign languages, and worked for a foreign company \citep[436]{zhang2005}. Zhang examined four phonetic variables, one of which was the realisation of a neutral tone as a full tone, a variable associated with dialects spoken outside Mainland China. Another variable she investigated was rhoticisation in syllable final position, a variable that was associated with the local Beijing dialect. Zhang found that only yuppies used the full tone variant; there were no examples of the state professionals using it. The government officials on the other hand were more likely to use the rhotic variable than were the yuppies. The linguistic variation observed in the workers' speech co-varied with a number of non-linguistic factors such as clothing and choice of music. The linguistic variation was a part of a larger picture, with speakers actively manipulating the range of stylistic tools that was at their disposal.

%In a study examining prosodic variation in the speech of three men, \citet{podesva2011} found that the speakers used different intonation contours in different types of interactions. The contours were manipulated to construct individual styles that were appropriate for a particular situation. For example, Heath, a doctor, used a greater number of high rising terminals when meeting with a patient than when at a social situation with friends. Podesva argues that the high rises helped to construct Heath's `caring doctor' persona. In contrast, while at a barbecue with friends, the same individual, Heath, used falling tones that were acoustically-extreme (tokens found to be outliers due to a larger pitch excursion than other tokens).  Podesva argued that these extreme falls contributed to Heath's persona as a diva.  Here, both frequent use of an intonational pattern (high rising terminals) and acoustically-extreme tokens carried social meaning and were used to construct a speaker's social identity in different interactional contexts.

Studies in the Third Wave demonstrate how speakers manipulate linguistic and non-linguistic factors in creating and exhibiting their style. Whereas studies in the First and Second Waves treat linguistic variables as indexed to either broad or local social categories, studies in the Third Wave investigate the social meaning of variables and how these variables contribute to an individual's persona. 

Much of the work regarded as Third Wave includes techniques used in the First and Second Waves, including the investigation of covariation between linguistic variables and social categories observed in a speech community. The work presented in this book employs multiple approaches; the role of the individual is discussed in \sectref{theindividual} and the relationship between a linguistic variable and a speaker's social grouping is discussed in \sectref{sec:prodresults}. While investigating stance and style is important to aid in the understanding of why many speakers use linguistic variants associated with a larger social group to which they belong, it does not make the examination of the relationship between variants and larger groups irrelevant. In fact, I would argue that macro-level (e.g., First Wave) variation informs style-making just as style-making is the vehicle through which macro-level variation arises.


%	 							     - 3rd wave: styles, rather than variables, are directly associated with identity categories and explores contributions of variables to styles.  Views variables as located within layered communities.  Examines not just variables such as changes in progress but any ling material that serves as a stylistic marker.  Focus on construction of personae.
  



Work in all three waves displays how phonetic variation is not merely ``noise'' but is meaningful and is a part of a speaker's communicative competence, the competence required by a speaker in order to communicate effectively \citep{hymes1972}. In order to be manipulated in such a systematic manner, the relationships must be stored in speakers' minds and accessed during speech production. Like linguistically-conditioned variation, socially-conditioned variation contributes to linguistic structure and is a reflection of a speaker's (not necessarily conscious) knowledge about language. But if this information is stored, it might also be expected to influence speech processing. In \sectref{sec:perception}, I discuss results from experimental work demonstrating that an individual's knowledge of sociophonetic trends from production does in fact influence speech perception. In the next section, I discuss studies that have used acoustic analysis to investigate a speaker's linguistic competence of nuanced variables.


\section{Gradience and acoustic analysis}\label{sec:acoustic}
Most formal phonological theories, such as those based on features or constraints, were not developed with gradience in mind. Some researchers working in these theories (e.g., \citealt{boersma1997}) have sought to incorporate methods of accounting for the probabilistic distribution of phonological variables. Still, few formal linguistic models can handle gradient phonetic data despite the fact that phonetic variables are not clear-cut categories but points along a multi-dimensional continuum. These dimensions include segment duration, vowel quality differences related to formant frequencies, the frequency range of aperiodic energy for fricated segments, and voice-quality features such as glottalisation and nasalisation; all of these can contribute to the overall quality of a token. In contrast with auditory analysis which necessarily treats variants as points in auditory/acoustic space, acoustic analysis allows investigation of gradient variables, such as duration, as well as variables where differences between the realisations are extremely subtle and therefore difficult to conduct auditory analysis on.   

Sociophonetic work which has used laboratory techniques to examine variation (e.g., \citealt{labov2001,labov2005}) has overwhelmingly focused on vowels, most often measuring the midpoint in the first and second formants (F1 and F2). When plotted on an F1-F2 graph, the measurements provide an idea of the height and backness of a token for a particular speaker relative to other variants produced by that speaker \citep[182-184]{petersonbarney1952}. \citet[466-497]{labov2001} plotted variables this way to demonstrate how different factors influence sounds undergoing change. Although acoustic analysis is more time-consuming than auditory analysis, it more accurately reflects the distribution of variables in acoustic space and demonstrates how phonetic variation is both systematic and gradient.  

Consonants can also differ depending on a combination of phonological and social factors. Most sociophonetic studies examining consonantal variation use auditory analysis. But as with vowels, some of the differences in realisations are nuanced, lending themselves to investigation by laboratory methods. Sociophonetic research that has conducted acoustic analysis on consonants includes work by \citet{haymaclagan2010}, \citet{dochertyfoulkes1999}, and \citet{foulkesdochertywatt2005}.

\citet{haymaclagan2010} investigated the relationship between \textipa{/r/} intrusion and social factors. They found that male speakers and New Zealanders from lower socioeconomic backgrounds were more likely to produce intrusive /r/ than females and New Zealanders from higher socioeconomic backgrounds. They also investigated the amount of constriction of the /r/, where a lower F3 value is impressionistically more /r/-like. Investigating only those tokens that were identified as having intrusive /r/, they found that participants from lower socioeconomic groups were more likely to produce the /r/ with a lower F3. The likelihood of intrusive /r/ depends on the social characteristics of the speaker and so does the degree of the constriction when producing the /r/. 

In addition to examining gradience, acoustic analysis provides a way to investigate highly nuanced phonetic variation. \citet{dochertyfoulkes1999} uncovered phonologically and so\-cially-condi\-tioned variation among realisations of pre-pausal and intervocalic /\textipa{t}/, variation that was so subtle that it had been overlooked by researchers conducting auditory analysis on similar tokens. They also found that children as young as two already exhibit the socially-conditioned variation in \textipa{/t/} realisation \citep{foulkesdochertywatt2005}. These findings provide evidence that individuals adopt socially-meaningful variables, even when differences between variants are extremely nuanced and difficult to perceive. This raises questions regarding the nature of the phonetic information that is stored in the mind: how detailed is it?

The work discussed in this section demonstrates the benefits of using acoustic analysis in sociophonetic investigations of both vowels and consonants. The results have important theoretical implications, providing support for probabilistic models of speech production and evidence that stored representations of phonetic information are acoustically detailed. They also raise questions about the relationship between an individual's production and perception: how is it possible that speakers produce socially-appropriate variants when differences between the variants are difficult to perceive without the aid of voice-analysis software? In the following section, I discuss work that aims to shed light on this question through the examination of the relationship between phonetic information and social characteristics in speech perception.


\section{Experimental sociolinguistics}\label{sec:perception}
Perception studies have yielded insights into how phonetic variation is stored in the mind through exploring the effects of non-lin\-guis\-tic information on speech processing. The research described in this section provides evidence that the social characteristics attributed to the speaker can influence how phones are perceived. This suggests that phonetic representations are indexed to non-lin\-guis\-tic information and that this non-lin\-guis\-tic information is accessed during speech pro\-cessing \citep{strand1999,campbellkibler2007,dragerunderrev}. Additionally, given the subtle phonetic differences between variants, the results provide evidence that the phonetic representation contains rich detail that previously was assumed to be filtered out during speech perception, storing only an abstracted form in the mental representation.

For example, the focus of the aperiodic energy of the alveolar fricative \textipa{/s/} is higher than for the palatal fricative \textipa{/S/} within the speech of a single individual. The acoustic boundary between \textipa{/s/} and \textipa{/S/} tends to be higher for females than males. This means that it is possible for a token of \textipa{/s/} produced by a male to have its turbulence focused in a similar frequency range as a female's token of \textipa{/S/}. In an experiment where video clips of men and women were matched with gender-ambiguous tokens from a \textipa{/s/} - \textipa{/S/} continuum, \citet{strand1999,strand2000} found that participants were more likely to perceive a token as \textipa{/S/} if shown a video of a female. In other words, the same fricative was perceived differently depending on the face with which it was paired. These results provide evidence that perceivers attribute social characteristics to a speaker and then use this information to help identify sounds produced by that speaker.

There is evidence that the perception of phonetic variables can also be affected by other social characteristics at\-tri\-but\-ed to the spea\-ker, including dialect area \citep{niedzielski1999,haynolandrager2006}, socioeconomic status \citep{haywarrendrager2006}, age \citep{haywarrendrager2006,drager2006,dragerunderrev}, and ethnicity \citep{staumcasasanto2010}. The centring diphthongs \textipa{/i@/} and \textipa{/e@/}, as in the words \textit{near} and \textit{square}, are undergoing a merger in NZE. This change has been led by members of lower socioeconomic groups; while some New Zealanders maintain the distinction, the diphthongs are merged in the speech of many New Zealanders who are young and/or members of lower socioeconomic groups. Using photographs to manipulate the perceived socioeconomic status and age of speakers in a perception experiment, \citet{haywarrendrager2006} found that participants' accuracy at identifying distinct tokens of the diphthongs depended on the social characteristics of the person in the photograph. Likewise, \citet{dragerunderrev} found that the age of the person in a photograph could influence perception of variants undergoing a chain shift in progress. Results from both of these studies provide further evidence that individuals access stored social information attributed to a speaker during speech perception and that this social information can affect how sounds are perceived.

In both Detroit and Canada, speakers produce variants of the diphthong /\textipa{au}/, as in the word \textit{mouth}, with a raised nucleus. Speakers from Detroit associate this variant with Canadians and are not aware that they also produce raised variants. \citet{niedzielski1999} conducted an experiment where participants were asked to match a vowel from natural speech to one from a synthesised vowel continuum ranging from raised variants to standard American English variants. She found that participants were more likely to respond with a raised token from the continuum if they were in the condition where \textit{Canada} appeared at the top of the response sheet than if they were in the condition where \textit{Michigan} was at the top of the response sheet. Niedzielski argues that participants shifted in their perception due to their expectations regarding the speaker's dialect area. In New Zealand, \citet{haynolandrager2006} found similar results in an experiment that was based on Niedzielski's paradigm and manipulated whether `New Zealand' or `Australia' was written at the top of the answersheet. In contrast to the variable in Niedzielski's study, the target vowel \textipa{/I/} was one with different realisations in the two dialects. While participants in the Australian condition were more likely to respond with an Australian token from the continuum than were participants in the New Zealand condition, all but one of the participants indicated that they in fact knew that the voice was a New Zealander. \citet{haynolandrager2006} argue that instead of expectations regarding a speaker's dialect area affecting performance on the task, the mere mention of another dialect area was enough to orient perception toward that dialect. 

The experiments outlined above investigate the extent to which speech perception can be affected by social characteristics that are either attributed to a speaker or triggered from exposure to a related stimulus. Another area of inquiry provided by experimental methodologies is an investigation of the degree to which phonetic cues in the stimulus and the participants' previous experience affect what social characteristics are attributed to the speaker. 

For example, \citet{clopperpisoni2004} conducted an experiment in which they played participants clips of speech produced by speakers from different parts of the US and participants were asked to indicate the regional origin of the speakers. They found that participants who had not lived in a dialect area were less accurate at identifying the dialect than participants who had lived there. In other words, accuracy on the task depended on the participants' prior exposure to the different dialects. 

\citet{campbellkibler2007} conducted a relevant and influential study in which she played groups of participants clips of speech and asked them to comment on the speakers (e.g., \textit{What can you tell me about Jason? Where do you think he's from?}). There were two experimental conditions. The clips of speech used in the conditions were identical except that word-final nasals were spliced so that in one condition the alveolar nasal [\textipa{n}] occurred in a word (e.g., \textit{fishin'} in the \textit{-in} guise) and in the other condition the velar nasal [\textipa{N}] occurred in that word (e.g., \textit{fishing} in the \textit{-ing} guise). Although all other aspects of the utterances were identical, speakers were more likely to be rated as educated and articulate when in the \textit{-ing} guise than when in the \textit{-in} guise. But the variable did not affect the perception of social characteristics equally for all voices; participants were more likely to identify one speaker in particular as gay, especially when in the \textit{-ing} guise. These results provide evidence that even slight shifts in phonetic realisations can influence what social characteristics are attributed to a speaker and that interpretations of speaker identity are based on a combination of multiple phonetic cues that are present in the signal; the meaning of a single variable can change when other socially-meaningful phonetic cues are inherent in the signal.

Taken together, results from sociophonetic perception experiments provide evidence that non-linguistic information attributed to a speaker is accessed during perception and can affect how sounds are perceived. In the following section, I discuss recent work investigating the relationship between phonetic variation, token frequency, and the lemma.

\section{Laboratory phonology}
In addition to exploring the link between phonetic variants and identity construction, the work presented in this book investigates current questions of interest within the scope of what is sometimes referred to as experimental or laboratory phonology. Laboratory phonology uses empirically-based methods to test and develop linguistic models of speech production and perception. Though phonetics and phonology remain a central focus of researchers working in this field, much of the work investigates how these prelexical levels influence the production and perception of other aspects of the grammar, including syntax \citep{haybresnan2006} and the lexicon \citep{bybee2002,gahl-thyme}. This book explores questions surrounding the frequency of a lexical item and the relationship between phonetic information and the lemma during speech processing.
 
	
	\subsection{Token frequency}\label{sec:frequency}
Researchers have noted a relationship between phonetic reduction and token frequency \citep{bybee2001,zipf1929} and there is some empirical evidence that such a relationship does indeed exist \citep{aylettturk2004,bakerbradlow2009,belletal2009}. There is also evidence that more frequent words are more likely to contain centralised vowels \citep{aylettturk2006,munsonsolomon2004}, which -- if centralisation is viewed as phonetic reduction -- also supports this argument. \citet{bybee2002} argues that reductive phonetic change exhibits lexical diffusion (the sound change occurs in some words before others) and that the most frequent lexical items are the first to undergo change. She outlines an array of work exemplifying how intervocalic \textipa{/D/} deletion in Spanish as well as t/d deletion and vowel reduction and deletion in English are linked to word frequency; reduction and deletion are more likely to occur in high frequency words than in low frequency words.\footnote{\citet{bybee2002} treats words that are observed in corpus data fewer than 35 times per million words as low-frequency.} There is also evidence that a speaker's vowel space is influenced by token frequency \citep{munson2007} and that, in tonal languages, there is a relationship between token frequency and the overall F0 and tone dispersion within that word \citep{zhaojurafsky2007}.\footnote{One of the few sociophonetic studies to include token frequency in the analysis was conducted by \citet{oprah1999}, who found that both lexical frequency and the ethnicity of the referee (the person being discussed) predicted \textipa{/ai/} monophthongisation in the speech of the television personality, Oprah Winfrey. The social effect of the referee was stronger than the effect of token frequency.} Like the sociophonetic work described earlier, this work on the effects of token frequency demonstrates how language is probabilistic rather than categorical; the ``messiness'' of \textit{parole} is far more structured than was previously believed.


	\subsection{Lemmas, lexemes, and phonetic detail}
	
The level at which frequency-based information is stored is not yet clear. Previous research differentiates between a lemma (a syntactically/semantically defined entry) and a lexeme (a wordform entry that specifies, for example, segmental information) \citep{bock1995}. There is evidence that phonetic variation not only occurs across words with different wordforms but that polysemes and ``homophones'', such as \textit{time} and \textit{thyme}, can have different realisations and these realisations can be predicted by the lemma's frequency \citep{gahl-thyme,jurafskyetal2002}. While some of the variation attributed to token frequency may instead be a function of how predictable a word is given its position in a sentence \citep{jurafskyetal2002}, \citet{gahl-thyme} found that, over and above effects from contextual predictability, words of homophone pairs can differ in regard to their durations: the more frequent word in the pair is more reduced. This suggests that the lemma (in addition to the lexeme) is a locus of token frequency information. 

If social factors are shown to influence the realisations of different words that share a wordform, it would suggest that either phonetic information is indexed to the lemma or that the lemmas do not share a single lexeme-based level of representation and the phonetic detail is indexed to their separate lexeme representations. If lemma-level representations are indexed to an additional representation where phonetic information is available, that information is inaccessible during a tip-of-the-tongue moment. \nocite{johnson1997}\nocite{pierrehumbert2001}\nocite{pierrehumbert2006} In this book, I do not attempt to tease apart these two possibilities and use \textit{lemma-based variation} for both.

							

 \section{Multiple methodologies}
The work presented in this book draws on insights gained from the research discussed in this chapter, combining the various methods and research questions within a single study with the aim of unifying all results within a model of speech production and perception. Ethnography, speech perception experiments, and acoustic analysis were used in order to take advantage of the benefits of each. Through ethnography, I was able to become familiar with the speakers and come to understand their individual styles and stances. Through conducting acoustic analysis on their speech, I was able to investigate subtle differences in realisations of tokens. And through conducting speech perception experiments with participants who were the same individuals who took part in the ethnographic portion of the study, I was able to test the effect of phonetic cues on the attribution of social information during speech perception. Additionally, the qualitative data collected during the ethnographic portion of the study helped to interpret the results from the perception experiments, further exemplifying the benefits of employing multiple methodologies.

The work in this book also investigates questions of interest outside the scope of sociolinguistics. For example, the frequency counts in all of the work described in \sectref{sec:frequency} were based on text-based corpora or spoken corpora from a multitude of different speakers and identical token frequency counts and lemma probabilities were used to examine effects across all speakers. However, the cognitive mechanisms to which these effects are attributed also predict an effect of speaker-specific token frequency and speaker-specific lemma-probability; if an individual speaker uses a lexical item more often in a given context, reductive phonetic changes such as those outlined by \citet{bybee2002} should be most advanced in that lexical item for that individual speaker. This hypothesis is tested in the production results described in \chapref{ch:prod}. Likewise, the lemma-based phonetic variation described by \citet{gahl-thyme} raises the question of whether perceivers can distinguish between auditory tokens of lemmas that share a wordform. The experiments presented in \chapref{ch:perc} address this question. Ultimately, the work in this book investigates identity construction, gradience, lemma probabilities, and the relationship between phonetic and lemma-based information. The findings are used to inform the model of speech production, perception and identity construction discussed in \chapref{ch:disc}. 

In the following chapter, I describe Selwyn Girls' High through a description of my experiences from the year I spent there. As the work described in \sectref{sec:waves} demonstrates, speakers' social characteristics and styles are complex as is the correlation between these styles and the phonetic variables produced. I ask that readers take the time while reading \chapref{ch:ethnography} to reflect on what life at Selwyn Girls' High was like and to recognise that while most of the girls belong to certain groups, each girl is a unique individual. Through investigating individuals and how they construct their identities and through investigating variation not only in their production but also in their perception of variables, I aim to provide further evidence that the observed variation and the indexical meanings are fundamental aspects of what constitutes a speaker's linguistic competence.


The ethnographic portion of this study was conducted at Selwyn Girls' High (SGH) in 2006 with the aim of becoming familiar with individuals at the school, determining what, if any, social categories were relevant for the girls, and identifying different styles and stances that were present at the school. I was especially interested in how different individuals constructed their social identities through the manipulation of both linguistic and non-linguistic variables. As will be discussed in \chapref{ch:prod}, some phonetic variation at the school appears to be linked to the girls' active construction of their social personae. 

In the following chapter I describe different experiences I had while at Selwyn Girls' High. Although I write from my point of view (and, in fact, start the narrative from my point of view), I have tried to focus the attention on the students rather than myself so that the reader may appreciate the richness of their lives and understand those aspects of life that the girls considered important. These are real people, with real frustrations and real excitement. But as explained by Narayan, we as ethnographers ``do not speak from a position outside `their' worlds, but are implicated in them'' \citet[676]{narayan1993}. Any results are only ``true'' insofar as they are understood in relation to ourselves being implemented within the reality of the speech community we are trying to describe. Additionally, findings should be interpreted within the context of our biased observations. We are not objective; our presence and previous biases are inseparable from ourselves. Therefore, I have tried to remind the reader throughout the text that this is only my story, my ``truth'', of the situation at Selwyn Girls' High and I apologise to the girls for presenting them in a way that reflects at best only a part of who they are. Still, though it fails to describe the girls entirely, I hope it reflects a part of each of them, however incompletely. 

%\newpage
%\thispagestyle{empty}
%\mbox{}

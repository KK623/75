\section*{Acknowledgements}

\noindent This book is a slightly altered version of my Ph.D. dissertation, which I completed with the insightful and generous help from the wonderful mentors I had while at the University of Canterbury.  First and foremost, I would like to thank Jen Hay.  I can't begin to express how indebted I am to her.  She is my mentor, my role model, and my good friend and it has been an incredible honour to work with her.  Through knowing Jen, I have come to understand the kind of teacher, researcher, and mentor that I would like to be.  I am also extremely grateful to Alex D'Arcy, my associate supervisor, for her valuable critiques of my work and her wonderfully caring and supportive nature.  Additionally, I would like to thank Christian Langstrof, Anita Szakay, Margaret Maclagan, Elizabeth Gordon, Heidi Quinn, and Jeanette King.  I would also like to thank Abby Walker, who I have had countless academic discussions with and who has been there for me on a personal level more times than I can count.

After completing the ethnographic portion of this study, I had the opportunity to spend time as a visiting student at two overseas universities: Stanford University and the University of Oxford.  While at Stanford, I had the pleasure of working with Penny Eckert, with whom every discussion resulted in a new insight.  I am also grateful to John Coleman for allowing me to work at the Oxford Phonetics Lab, where I completed the majority of the acoustic analysis that is presented in this book.  

I am extremely appreciative of the helpful comments made by Margaret Maclagan, Felcity Cox, Jane Stuart-Smith, Lauren Hall-Lew, Laura Staum Casasanto, Rebecca Greene, Gerry Docherty, Paul Foulkes, Benjamin Munson, `\=Oiwi Parker Jones, and Keith Johnson.  I would also like to thank Gerry Docherty, who served as a reviewer for this book and whose comments and suggestions were especially valuable.

Thank you to Carolyn Morris, Martin Fuchs, and Norma Men\-doza-Den\-ton for providing support and advice regarding the ethnographic portion of the study.  I would like to thank Robert Fromont for his (continued) development of ONZEMiner, some of which he developed specifically for the requirements of this work.  And I would like to thank the University of Canterbury for funding this research through the University of Canterbury Targeted Scholarship.

I could never have completed this project without the help of my friends and family.  Never once have I doubted the love and support of my parents, Chris and Charlene Drager, or my brother, Blake.  And I'm not sure I would have been able to finish this work without the support of my friends, Emma Parnell and Alice Murphy.  Thank you. 

I would also like to thank Selwyn Girls' High for allowing me to conduct an ethnography at their school.  And last, but certainly not least, I would like to express my sincere gratitude to the girls of Selwyn Girls' High, who have given me so much more than I was able to give to them.